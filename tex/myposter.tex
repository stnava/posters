%\documentclass[portrait,final,a0paper]{tex/baposter}
\documentclass[paperwidth=48in,paperheight=48in,portrait,final]{tex/baposter}
% Usa a4shrink for an a4 sized paper.

\tracingstats=2

\usepackage{calc}
\usepackage{graphicx}
\usepackage{amsmath}
\usepackage{amssymb}
\usepackage{tex/relsize}
\usepackage{multirow}
\usepackage{bm}

\usepackage{graphicx}
\usepackage{multicol}


\usepackage{pgfbaselayers}
\pgfdeclarelayer{background}
\pgfdeclarelayer{foreground}
\pgfsetlayers{background,main,foreground}

\usepackage{times}
\usepackage{helvet}
%\usepackage{bookman}
% \usepackage{palatino}

\newcommand{\captionfont}{\footnotesize}

\selectcolormodel{cmyk}

\graphicspath{{images/}}

%%%%%%%%%%%%%%%%%%%%%%%%%%%%%%%%%%%%%%%%%%%%%%%%%%%%%%%%%%%%%%%%%%%%%%%%%%%%%%%%
%%%% Some math symbols used in the text
%%%%%%%%%%%%%%%%%%%%%%%%%%%%%%%%%%%%%%%%%%%%%%%%%%%%%%%%%%%%%%%%%%%%%%%%%%%%%%%%
% Format 
\newcommand{\Matrix}[1]{\begin{bmatrix} #1 \end{bmatrix}}
\newcommand{\Vector}[1]{\Matrix{#1}}
\newcommand*{\SET}[1]  {\ensuremath{\mathcal{#1}}}
\newcommand*{\MAT}[1]  {\ensuremath{\mathbf{#1}}}
\newcommand*{\VEC}[1]  {\ensuremath{\bm{#1}}}
\newcommand*{\CONST}[1]{\ensuremath{\mathit{#1}}}
\newcommand*{\norm}[1]{\mathopen\| #1 \mathclose\|}% use instead of $\|x\|$
\newcommand*{\abs}[1]{\mathopen| #1 \mathclose|}% use instead of $\|x\|$
\newcommand*{\absLR}[1]{\left| #1 \right|}% use instead of $\|x\|$

\def\norm#1{\mathopen\| #1 \mathclose\|}% use instead of $\|x\|$
\newcommand{\normLR}[1]{\left\| #1 \right\|}% use instead of $\|x\|$

%%%%%%%%%%%%%%%%%%%%%%%%%%%%%%%%%%%%%%%%%%%%%%%%%%%%%%%%%%%%%%%%%%%%%%%%%%%%%%%%
% Multicol Settings
%%%%%%%%%%%%%%%%%%%%%%%%%%%%%%%%%%%%%%%%%%%%%%%%%%%%%%%%%%%%%%%%%%%%%%%%%%%%%%%%
\setlength{\columnsep}{0.7em}
\setlength{\columnseprule}{0mm}


%%%%%%%%%%%%%%%%%%%%%%%%%%%%%%%%%%%%%%%%%%%%%%%%%%%%%%%%%%%%%%%%%%%%%%%%%%%%%%%%
% Save space in lists. Use this after the opening of the list
%%%%%%%%%%%%%%%%%%%%%%%%%%%%%%%%%%%%%%%%%%%%%%%%%%%%%%%%%%%%%%%%%%%%%%%%%%%%%%%%
\newcommand{\compresslist}{%
\setlength{\itemsep}{1pt}%
\setlength{\parskip}{0pt}%
\setlength{\parsep}{0pt}%
}


%%%%%%%%%%%%%%%%%%%%%%%%%%%%%%%%%%%%%%%%%%%%%%%%%%%%%%%%%%%%%%%%%%%%%%%%%%%%%%
%%% Begin of Document
%%%%%%%%%%%%%%%%%%%%%%%%%%%%%%%%%%%%%%%%%%%%%%%%%%%%%%%%%%%%%%%%%%%%%%%%%%%%%%

\begin{document}

%%%%%%%%%%%%%%%%%%%%%%%%%%%%%%%%%%%%%%%%%%%%%%%%%%%%%%%%%%%%%%%%%%%%%%%%%%%%%%
%%% Here starts the poster
%%%---------------------------------------------------------------------------
%%% Format it to your taste with the options
%%%%%%%%%%%%%%%%%%%%%%%%%%%%%%%%%%%%%%%%%%%%%%%%%%%%%%%%%%%%%%%%%%%%%%%%%%%%%%
% Define some colors
\definecolor{silver}{cmyk}{0,0,0,0.3}
\definecolor{yellow}{cmyk}{0,0,0.9,0.0}
\definecolor{reddishyellow}{cmyk}{0,0.22,1.0,0.0}
\definecolor{black}{cmyk}{0,0,0.0,1.0}
\definecolor{darkYellow}{cmyk}{0,0,1.0,0.5}
\definecolor{darkSilver}{cmyk}{0,0,0,0.1}

\definecolor{lightyellow}{cmyk}{0,0,0.3,0.0}
\definecolor{lighteryellow}{cmyk}{0,0,0.1,0.0}
\definecolor{lightestyellow}{cmyk}{0,0,0.05,0.0}

% http://www.creativecolorschemes.com/resources/free-color-schemes/blue-tone-color-scheme.shtml

\definecolor{colorA}{cmyk}{1,0.25,0.1,0.5}
\definecolor{colorB}{cmyk}{1,0.3,0.05,0.2}
\definecolor{colorC}{cmyk}{0.6,0,0.05,0}
\definecolor{colorD}{cmyk}{0.95,0.6,0.05,0.15}
\definecolor{colorE}{cmyk}{0.6,0.3,0.05,0.05}
\definecolor{colorF}{cmyk}{0.4,0.15,0,0}
\definecolor{colorf}{cmyk}{0.1,0.0375,0,0}
\definecolor{colorG}{cmyk}{0.75,0.3,0.05,0.1}
\definecolor{colorH}{cmyk}{0.65,0.15,0,0.05}
\definecolor{colorI}{cmyk}{0.3,0.05,0,0}
\definecolor{colorJ}{cmyk}{0.65,0,1,0}
\definecolor{colorK}{cmyk}{0,0.5,1,0}
\definecolor{colork}{cmyk}{0,0.125,0.25,0}
\definecolor{colorL}{cmyk}{0.65,0.8,0,0}
\definecolor{colorM}{cmyk}{0.7,0.4,0.2,0.6}
\definecolor{colorN}{cmyk}{0.3,0.15,0.1,0.3}
\definecolor{colorn}{cmyk}{0.075,0.0375,0.025,0.075}
\definecolor{colorO}{cmyk}{0.25,0.05,0.1,0.15}

%%
\typeout{Poster Starts}
\background{
  \begin{tikzpicture}[remember picture,overlay]%
    \draw (current page.north west)+(-2em,2em) node[anchor=north west] {\includegraphics[height=1.1\textheight]{silhouettes_background}};
  \end{tikzpicture}%
}

\newlength{\leftimgwidth}
\begin{poster}%
  % Poster Options
  {
  % Show grid to help with alignment
  grid=no,
  % Column spacing
  colspacing=1em,
  % Color style
  bgColorOne=colorf,
  bgColorTwo=white,
  borderColor=colorK,
  headerColorOne=colork,
  headerColorTwo=white,
  headerFontColor=colorM,
  boxColorOne=white,
  boxColorTwo=colorC,
  % Format of textbox
  textborder=roundedleft,
%  textborder=rectangle,
  % Format of text header
  eyecatcher=yes,
  headerborder=closed,
  headerheight=0.11\textheight,
  headershape=roundedright,
  headershade=shade-lr,
  headerfont=\Large, %Sans Serif
  boxshade=plain,
%  background=shade-tb,
  background=plain,
  linewidth=2pt
  }
  % Eye Catcher
  {\includegraphics[width=10em]{figures/picslsquare}} % No eye catcher for this poster. (eyecatcher=no above). If an eye catcher is present, the title is centered between eye-catcher and logo.
  % Title
  { %Sans Serif
  %\bf% Serif
~~~~~~~~~~~Multivariate neuroimaging substrates for five cognitive domains
in aging and focal neurodegeneration
  \vspace{0.1em}}
  % Authors
  { %Sans Serif
  % Serif
  ~~~~~~~~~~~B. Avants, P. A. Cook, J. T. Duda, D. Libon$^\dagger$,
  C. McMillan, J. C. Gee, M. Grossman
  \vspace{0.2em}
% Affiliations
{\normalsize \newline Depts. of Radiology and Neurology~~~~~~~~~~~~~~~~~~~~~~~~~~~~~~~~~~~~~~Dept. of Neurology$^\dagger$ \\
  University of Pennsylvania, Philadelphia, PA, USA~~~~~~~~~~~Drexel University, Philadelphia, PA, USA
  }
  
  }
  % University logo
  {% The makebox allows the title to flow into the logo, this is a hack because of the L shaped logo.
%    \makebox[8em][r]{%
      \begin{minipage}{16em}
        \hfill
%        \includegraphics[height=2em]{msrlogo}
        \includegraphics[height=7.0em]{figures/pennshield}
      \end{minipage}
%    }
  }

  \tikzstyle{light shaded}=[top color=baposterBGtwo!30!white,bottom color=baposterBGone!30!white,shading=axis,shading angle=30]

  % Width of left inset image
     \setlength{\leftimgwidth}{0.78em+8.0em}

%%%%%%%%%%%%%%%%%%%%%%%%%%%%%%%%%%%%%%%%%%%%%%%%%%%%%%%%%%%%%%%%%%%%%%%%%%%%%%
%%% Now define the boxes that make up the poster
%%%---------------------------------------------------------------------------
%%% Each box has a name and can be placed absolutely or relatively.
%%% The only inconvenience is that you can only specify a relative position 
%%% towards an already declared box. So if you have a box attached to the 
%%% bottom, one to the top and a third one which should be in between, you 
%%% have to specify the top and bottom boxes before you specify the middle 
%%% box.
%%%%%%%%%%%%%%%%%%%%%%%%%%%%%%%%%%%%%%%%%%%%%%%%%%%%%%%%%%%%%%%%%%%%%%%%%%%%%%
    %
    % A coloured circle useful as a bullet with an adjustably strong filling
    \newcommand{\colouredcircle}[1]{%
      \tikz{\useasboundingbox (-0.2em,-0.32em) rectangle(0.2em,0.32em); \draw[draw=black,fill=baposterBGone!80!black!#1!white,line width=0.03em] (0,0) circle(0.18em);}}

%%%%%%%%%%%%%%%%%%%%%%%%%%%%%%%%%%%%%%%%%%%%%%%%%%%%%%%%%%%%%%%%%%%%%%%%%%%%%%
  \headerbox{Introduction}{name=introduction,column=0,row=0}{
%%%%%%%%%%%%%%%%%%%%%%%%%%%%%%%%%%%%%%%%%%%%%%%%%%%%%%%%%%%%%%%%%%%%%%%%%%%%%%

Neuropsychological measures detect high-level cognitive function that
emerges from complex, large-scale neural
networks that include multiple gray matter (GM) regions integrated by
white matter (WM) projections. The clinically-validated Philadelphia
Brief Assessment of Cognition (PBAC) \cite{Libon2007} examines several
cognitive domains, including executive functioning/working memory
(exe), language (lang), visuospatial skills (vs), visual/verbal
episodic memory (mem) and social comportment/behavior (behav). We use
the open-source sparse canonical correlation analysis for neuroimaging
(SCCAN) software to {\bf cross-validate} putative relationships
between the neural substrate and cognition.   We thus relate three quantitative measures: the PBAC, GM density (GMD) derived from T1 MRI, and the fractional anisotropy (FA) of WM derived from DTI.
  \vspace{0.3em}
 }

%%%%%%%%%%%%%%%%%%%%%%%%%%%%%%%%%%%%%%%%%%%%%%%%%%%%%%%%%%%%%%%%%%%%%%%%%%%%%%
  \headerbox{Methods}{name=methods,column=0,below=introduction}{
%%%%%%%%%%%%%%%%%%%%%%%%%%%%%%%%%%%%%%%%%%%%%%%%%%%%%%%%%%%%%%%%%%%%%%%%%%%%%%
\paragraph{Description of cohort: }
104 subjects had complete cognitive evaluations as well as Siemens
3.0T T1 and DTI collected at the University of Pennsylvania. The
cohort contained subjects diagnosed with Alzheimer's disease (AD;
n=23), behavioral- variant frontotemporal degeneration (bvFTD; n=31),
primary progressive aphasia (PPA) variants of FTD (n=24), 17 with
extrapyramidal motor disorders, and 9 elderly controls. We use
multiple different disorders only to obtain variance in cognition but
do not include clinical diagnosis in subsequent analyses. We
normalized and segmented \cite{Avants2011a} the DT and T1 images using
the ITK-based open-source ANTs \cite{Avants2011} and Pipedream software.
\newline
\paragraph{Statistical methods:} SCCAN minimizes the multiple comparisons problem while
maximizing voxel-level sensitivity by performing a whole brain to
cognition correlation that retains all of the voxels in the FA of WM
or GM probability data, and using them as predictors against the
cognitive battery and vice-versa until convergence. To provide spatial
specificity and interpretability, we enforce a positive relationship
between FA in WM and GMD that depends on 10\% of the GM and WM
respectively. No positivity or sparseness constraints are employed on
the PBAC variables. For all tests, significance was estimated via
permutation testing with 2000 trials. \cite{Avants2010b}
     \vspace{0.3em}
  }

%%%%%%%%%%%%%%%%%%%%%%%%%%%%%%%%%%%%%%%%%%%%%%%%%%%%%%%%%%%%%%%%%%%%%%%%%%%%%%
  \headerbox{Results}{name=results,column=0,span=1,below=methods}{
%%%%%%%%%%%%%%%%%%%%%%%%%%%%%%%%%%%%%%%%%%%%%%%%%%%%%%%%%%%%%%%%%%%%%%%%%%%%%%
\paragraph{Part 1-imaging \& cognition:} SCCAN identifies, for each PBAC domain, uniquely related WM regions. We perform the same type of study, independently, using GMD. The output from each run of SCCAN is the subset of WM / GM voxels that (as a set) relate most significantly to the cognitive domain of interest (e.g. language).
\newline
\newline
 We found significant associations between GM and each domain of
cognition consistent with putative neuroanatomical substrates at the
p<0.001 level. FA in WM was related to exec, soc, vs (p<0.05), weakly
related to memory (p<0.072) and unrelated to lang (p<0.89).
\newline
\paragraph{Part 2:}~We directly tested relationships between
 cognition-specific GMD and FA voxels derived from Part A. For
 example, we test if language-defined GMD voxels relate significantly
 to language-defined FA voxels, where SCCAN treats the voxels as a
 set.
\newline
\newline
 FA regions and GM regions were strongly related within each cognitive
domain (p<0.001) except for language where the relation was modest
(p<0.05). Fig 1 and 2 show the anatomic distribution of the
cognition-specific regions for both WM and GM for each cognitive
domain. Fig 3 shows the behav-based GMD and FA scatterplot.
\newline
\paragraph{Highlights:} Behav relates strongly to medial orbitofrontal cortex GM and genu in WM. Vs to occipital GM (cuneus) and occipital projections. Exec to insula, dorsolateral prefrontal, lateral orbitofrontal, and bilateral hippocampus GM regions and WM projections in the superior frontal lobe and the anterior callosum. Mem to GM insula, cuneus and posterior superior temporal gyrus, and WM fornix and superior longitudinal fasciculus.
  \vspace{0.3em}
  }



%%%%%%%%%%%%%%%%%%%%%%%%%%%%%%%%%%%%%%%%%%%%%%%%%%%%%%%%%%%%%%%%%%%%%%%%%%%%%%
  \headerbox{Cortical gray matter density and five cognitive domains}{name=workflow1,column=1,span=2}{
%%%%%%%%%%%%%%%%%%%%%%%%%%%%%%%%%%%%%%%%%%%%%%%%%%%%%%%%%%%%%%%%%%%%%%%%%%%%%%
  \vspace{0.3em}

\begin{center}
\begin{tabular}{cc}
 \includegraphics[height=80mm]{figures/ants} & \includegraphics[height=70mm]{figures/ants} \\
\end{tabular}
\end{center}


  }


%%%%%%%%%%%%%%%%%%%%%%%%%%%%%%%%%%%%%%%%%%%%%%%%%%%%%%%%%%%%%%%%%%%%%%%%%%%%%%
  \headerbox{White matter fractional anisotropy and five cognitive domains}{name=workflow2,column=1,span=2,below=workflow1}{
%%%%%%%%%%%%%%%%%%%%%%%%%%%%%%%%%%%%%%%%%%%%%%%%%%%%%%%%%%%%%%%%%%%%%%%%%%%%%%
  \vspace{0.3em}

\begin{center}
\begin{tabular}{cc}
\includegraphics[height=80mm]{figures/ants} & \includegraphics[height=75mm]{figures/ants} \\
\end{tabular}
\end{center}

  }



%%%%%%%%%%%%%%%%%%%%%%%%%%%%%%%%%%%%%%%%%%%%%%%%%%%%%%%%%%%%%%%%%%%%%%%%%%%%%
  \headerbox{Conclusions}{name=conclusions,column=1,span=2,below=workflow2}{
%%%%%%%%%%%%%%%%%%%%%%%%%%%%%%%%%%%%%%%%%%%%%%%%%%%%%%%%%%%%%%%%%%%%%%%%%%%%%
 {\bf Our results confirm putative brain-behavior associations.}  At
 the least, they suggest unique relationships between cognitive variation
 and large-scale GM-WM networks that vary uniquely with cognitive domain. Furthermore,
 these results suggest that SCCAN may enhance detection power over
 traditional univariate approaches.  In particular, the significance
 of GM and cognition relationships ($\approx$  p<0.001) far exceeds those of the FA
 cognition relationships ($\approx$ p<0.05).  Despite this, there is
 significant association between FA-cognition identified voxels and GM-cognition
 identified voxels ($\approx$  p<0.001).  
  }

%%%%%%%%%%%%%%%%%%%%%%%%%%%%%%%%%%%%%%%%%%%%%%%%%%%%%%%%%%%%%%%%%%%%%%%%%%%%%
  \headerbox{References}{name=references,column=1,span=2,below=conclusions}{
%%%%%%%%%%%%%%%%%%%%%%%%%%%%%%%%%%%%%%%%%%%%%%%%%%%%%%%%%%%%%%%%%%%%%%%%%%%%%
%\vspace{2mm}
%\begin{tabular}{cc}
%\includegraphics[height=20mm]{antslogo} &  \includegraphics[height=20mm]{antslogo}\\
%\vspace{-1mm}\\
%\end{tabular}

    \smaller

      \vspace{-0.4em}
      \renewcommand{\refname}{\vspace{-0.8em}}
      \bibliographystyle{abbrv}
      \bibliography{references}

%    \bibliographystyle{ieee}
%    \renewcommand{\section}[2]{\vskip 0.05em}
%      \begin{thebibliography}{1}\itemsep=-0.01em
%      \setlength{\baselineskip}{0.4em}
%      \bibitem{amberg07:nonrigid}
%        B.~Amberg, S.~Romdhani, T. Vetter.
%        \newblock {O}ptimal {S}tep {N}onrigid {ICP} {A}lgorithms for {S}urface {R}egistration
%        \newblock In {\em Computer Vision and Pattern Recognition 2007}
%      \bibitem{amberg08:recognition}
%        B.~Amberg, R.~Knothe, T. Vetter.
%        \newblock Expression Invariant Face Recognition with a 3D Morphable Model
%        \newblock In {\em Automated Face and Gesture Recognition 2008}
%      \end{thebibliography}
  }

 \headerbox{Software}{name=software,column=1,span=2,below=references}{
%%%%%%%%%%%%%%%%%%%%%%%%%%%%%%%%%%%%%%%%%%%%%%%%%%%%%%%%%%%%%%%%%%%%%%%%%%%%%
\vspace{2mm}
\begin{tabular}{cc}
\includegraphics[height=20mm]{figures/ants} &  \includegraphics[height=20mm]{figures/itkv4}\\
\vspace{-1mm}\\
\end{tabular}
}


\end{poster}

\end{document}
